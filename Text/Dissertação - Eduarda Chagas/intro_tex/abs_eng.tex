In recent years we have seen significant growth in the number of intelligent applications involving analysis, data mining, and classification.
With the increase in the complexity of the investigations, the need for simple, fast, and low computational approaches has become essential.
In the context of non-parametric analysis of time series, the use of the Bandt-Pompe symbolization methodology has become relevant.
The use of ordinal patterns formed by time-series elements when combined with the use of information theory descriptors proved to have a high power of characterization of the process underlying the dynamics of the data.

Among the descriptors, two of these for presenting complementary definitions have received a great prominence in the literature: Shannon's entropy, which in this context measures the degree of disorder in the distribution of ordinal patterns formed through the time series, and the statistical complexity, which on the other hand, represents the degree of structural dependence between the elements of the sequence.
Together, these features form the Complexity-Entropy plane, whose present work aims to highlight and solve its main gaps:
~(i) the absence of methods to build confidence regions and 
~(ii) the ambiguity in the formation of symbols caused by the lack of information on the amplitude of the elements.
In order to present alternative methods for the reported problems, we propose two solutions: a modification in the transition graph of ordinal patterns, the Weighted Amplitude Transition Graph, which performs the calculation of the weight of its edges using amplitude variation information between the symbols, and the HC-PCA, a method of generating empirical confidence regions on the plane.
To validate our proposals, applications in the context of remote sensing and analysis of white noise sequences were developed.

\textbf{Keywords:}: Bandt-Pompe Symbolization, Ordinal Patterns, Complexity-entropy Plane, Information theory.