\chapter{Case study: White Noise Confidence Regions}\label{chapter:confidenceRegions}

In this section, we will present the results obtained by the first proposal for the construction of empirical confidence regions in the Complexity-Entropy plane for white noise models.
Through these regions, we want to check if the randomness present in such PRNGs can be captured by the descriptors even with the use of short sequences.
As a consequence, we observed that the proposed methodology proved to be consistent and coherent, managing to capture the randomness of truly random sequences, reproducing the results obtained by generators previously analyzed in the literature, and proving to be robust in the addition of correlation structures.

\section{Introduction}\label{conf:intro}

Several works have used deterministic and pseudorandom sequences aiming at understanding the properties of the points they produce in the $H\times C$ plane.
\cite{GeneralizedStatisticalComplexityMeasuresGeometricalAnalyticalProperties} analyzed the logistic chaotic map and discuss the boundaries of the $H \times C$ plane.
\cite{De_Micco_2009} studied chaotic components in pseudorandom number generators.
\cite{DistinguishingNoiseFromChaos}  tackled the often hard problem of distinguishing chaos from noise.
\cite{DistinguishingChaoticStochasticDynamicsTimeSeriesMultiscaleSymbolicApproach} used a multi-scale approach to analyze the interplay between chaotic and stochastic dynamics.

%With the knowledge of the expected variability of such points, according to the underlying dynamics, we can make hypothesis tests for a wide variety of models.
%Results in this direction can be found in the literature.
%\cite{RandomNumberGeneratorsCausality} showed that the Entropy-Complexity plane ($H\times C$) is a good indicator of the results of Diehard tests for pseudorandom number generators.
%\cite{De_Micco_2008} assessed ways of improving pseudorandom sequences by their representation in this plane.

However, we were able to verify that the reported works carry out statistical studies using as a basis the descriptors of Information Theory.
Motivated by previous works, in this work, we advance the state-of-the-art providing the first test for white
noise points in the $H \times C$ plane. 
In this proposal, the input is a sequence of true random observations generated by a physical-based procedure. 
We obtain the confidence regions by performing an orthogonal projection of the data onto the space of principal components, thus eliminating the restrictions imposed by the bounded space of the Complexity-Entropy plane.
Our contributions can be summarized as follows:
\begin{itemize}
    \item We provide the first contribution in constructing a test in the Complexity-Entropy Plane: we provide confidence regions and $p$-values.
    \item We evaluate this test’s size by analyzing random sequences generated by physical procedures and pseudorandom generators (PRNGs).
    \item We verify the test’s power contrasting correlated noise time series.
\end{itemize}

\section{Experimental Settings}

We evaluated the performance of the proposed method in relation to a large set of random sequences provided by state-of-the-art pseudo-random number generators.
In this section, we present the settings of the parameters that we use as a reference, the true random physical generators used to calculate the empirical distribution, and descriptive analysis of representative points in relation to the confidence regions.

\subsection{Parameters Settings and Dataset}

We conducted an ablation study to identify the influence of the parameters $T$, $D$, and $\tau$ in the construction of empirical confidence regions.
We verified that the results involving the time delay parameter variation did not show significant differences in repeated experiments; therefore, in the sequel, we did no consider $\tau$ as a determining factor.
On the other hand, we found two relevant variables: 
the length of the sequence 
and the embedding dimension.
We, thus, employed the following factors:
\begin{itemize}
\item Sequence length $T\in\mathcal T=\{T = \num[scientific-notation=true]{e3}, \num[scientific-notation=true]{5 e4}\}$,
\item Embedding dimension $D\in\mathcal D=\{3, 4, 5, 6\}$.
\end{itemize}
and kept $\tau=1$, which is the most frequently used option.
The values of $D$ are within the range recommended in the literature~\citep{PermutationEntropyBandtPompe}.

Using this parametric space, we analyzed the different degrees of information captured by the ordinal patterns formed.
For the construction of the confidence regions presented, we used:
\begin{itemize}
	\item A set of \num{104596} points in the $H \times C$ plane, corresponding to sequences of length $T = 1000$, for each value of $D\in \mathcal D$, and
	\item  a set of \num{2093} points in the $H \times C$ plane, corresponding to sequences of length $T = 50000$, for each value of $D \in \mathcal D$.
\end{itemize}

We used the \texttt R platform \citep[][v.~4.0.3]{Rmanual} for data generation and analyses, and the \texttt{ggplot2} library \cite{ggplot2Wickman} for generating the plots.

\subsection{True Random Numbers}

Random numbers are used in many fields, from gambling to criptography, aiming to guarantee a secure, realistic or unpredictable behavior. 
Pseudo randomic results can be achieved by software in a deterministic way. 
But, some applications need actual random numbers (despite the somewhat elusive nature of actual randomness).
Randomness can be observed in unpredictable real world phenomena like cathodic radiation or atmospheric noise.
%In this study we used two sources of real random numbers. 
%The first is based on vacuum states to generate random quantum numbers described by \citeauthor{RNGVacuumStates}~(\citeyear{RNGVacuumStates}), the second one is based on atmospheric noise captured by a cheap radio receiver presented at \url{www.random.org}.

In this study we used two sources of random numbers, here called true random, both from physical phenomena observation and measurement.
The first is based on vacuum states to generate random quantum numbers, the setup consists of an ordinary laser source to generate a local oscillator (LO), a half-wave plate, a polarizing beamsplitter (BPS), and two balanced detectors working together adding or subtracting the photocurrents results in a quadrature measurement of the LO or vacuum state. 
The probability distribution of the vacuum state is binned into $2^n$ equal parts (bins of same size), than, assigning a fixed bit combination of length \textit{n} to each sample point in a given bin \citeauthor{RNGVacuumStates}~(\citeyear{RNGVacuumStates}). 
The second one is based on atmospheric noise captured by a cheap radio receiver, started as a gambling engine, the randomness comes from an ordinary radio receiver that has no filter for static unwanted sounds caused by atmospheric noise, but perfect for random purposes, developed over a distributed setup with some radios located at different geographical locations sending random bits to a cloud server who process data and hosts random.org, the history, and some other information could be found at \cite{RandomOrg}.
We used \SI{54e6}{4\byte} words from each physical generator, which approximately amounts \SI{200}{\mega\byte} of data.

In Fig.~\ref{fig:white-noise}, we show how the samples used to build the empirical confidence regions are arranged in the $H \times C$ plane.


\begin{figure}
    \centering
    \includegraphics[width=\linewidth]{Figures/Points-PDF.png}
    \caption{White noise samples considered during the construction of the proposed confidence regions.}
    \label{fig:white-noise}
\end{figure}

\section{Experimental Results and Analysis}\label{Sec:Results}

\subsection{Descriptive analysis of empirical confidence regions}

The regions used as a reference in this work are obtained through true random sequences, where we extract the empirical distribution of white noises in the Complexity-Entropy plane.
Tables~\ref{Tab:Points1k} and~\ref{Tab:Points50k} list the coordinates in the $H\times C$ plane
of the emblematic point $\bm P'$ and of the four points $\bm P_1,\bm P_2,\bm P_3,\bm P_4$ that define the confidence regions at \SI{90}{\percent}, \SI{95}{\percent}, \SI{99}{\percent}, and \SI{99.9}{\percent}, 
for $D=3,4,5,6$ and $N=1000,50000$, .
The points are presented counterclockwise, starting with the one with the largest complexity.

Fig.~\ref{fig:HC-PCA} shows the results for $T = 50000$ and $D = 3,6$ in the new principal components space, along with the quantiles of order $\SI{90}{\percent}$, $\SI{95}{\percent}$, $\SI{99}{\percent}$, and $\SI{99.9}{\percent}$.
We also show the projection of the $H \times C$ plane boundaries in this space, as well the median of each data set, the latter being represented as red dots.
The confidence regions exceed the $H \times C$ boundaries, but this issue does not compromise the test's size since no points can be observed outside such boundaries.

Fig.~\ref{fig:HC-PCA} also shows that the data are not evenly distributed among the axes of the first principal component.
They tend to concentrate close to the point that corresponds to $(1,0)$ in the $H\times C$ plane.
As we use order statistics to define the confidence regions, this issue is also of little relevance for our results.
Moreover, Fig.~\ref{fig:PCA-Hist} shows that such asymmetry diminishes when the embedding dimension $D$ increases.

\begin{figure*}
	\centering
	\subfigure[Points in the Principal Components plane for for $D=3$]{\includegraphics[width=.49\linewidth]{Figures/HC-PCA-Trozos-D3N50k}}
	\subfigure[Points in the Principal Components plane for for $D=6$]{\includegraphics[width=.49\linewidth]{Figures/HC-PCA-Trozos-D6N50k}}
	\caption{Representation of true random white noise sequences of length $T = 50000$ in the PCA space for $D = 3$ and $D = 6$, and the quantiles of $\SI{90}{\percent}$, $\SI{95}{\percent}$, $\SI{99}{\percent}$, and $\SI{99.9}{\percent}$.}
	\label{fig:HC-PCA}
\end{figure*} 

As we can see in Fig.~\ref{fig:PCA-Hist} in the new representation space produced by the PCA, the data are not evenly distributed among the axes of the first main component, maintaining the character presented in the $H \times C$ plane, since such points tend to be concentrated close to the point $(1, 0)$.

\begin{figure}
	\centering
	\includegraphics[width=\linewidth]{Figures/PCA-hist-50k}
	\caption{Histograms of the first principal component for $D=3,4,5,6$}
	\label{fig:PCA-Hist}
\end{figure}

\begin{table}[hbt]
	\caption{Coordinates in the $H\times C$ plane of the emblematic series and the points that define the confidence regions at \SI{90}{\percent}, \SI{95}{\percent}, \SI{99}{\percent}, and \SI{99.9}{\percent} for $D=3,4,5,6$ and $N=1000$}
	\label{Tab:Points1k}
	\centering
    \scalebox{0.78}{
	\begin{tabular}{rccccc}
		\toprule
		& & \multicolumn{4}{c}{$N = 1000$}\\ 
		\cmidrule(lr){3-6} 
		$D$ & Point & \SI{90}{\percent} & \SI{95}{\percent} & \SI{99}{\percent} & \SI{99.9}{\percent}\\ 
		\cmidrule(lr){3-3} 
		\cmidrule(lr){4-4} 
		\cmidrule(lr){5-5} 
		\cmidrule(lr){6-6} 
		$3$ & $\bm P'$ & \multicolumn{4}{c}{$(0.9992089,0.0007800)$}\\
		& $\bm P_1$ & $(0.9973334, 0.0025601)$ & $(0.9967311, 0.0031343)$ & $(0.9953009, 0.0045054)$ & $(0.9931825, 0.0065387)$\\
		& $\bm P_2$ & $(0.9974047, 0.0026304)$ & $(0.9968219, 0.0032238)$ & $(0.9954349, 0.0046375)$ & $(0.9933704, 0.006724)$\\
		& $\bm P_3$ & $(0.9999497, 0)$ & $(0.9999398, 0)$ & $(0.9999203, 0)$ & $(0.9998925, 0)$\\
		& $\bm P_4$ & $(1, \num[scientific-notation=true]{5.17e-05})$ & $(1, \num[scientific-notation=true]{6.12e-05})$ & $(1, \num[scientific-notation=true]{8.45e-05})$ & $(1, 0.0001104)$\\ 
		\midrule
		$4$ & $\bm P'$ & \multicolumn{4}{c}{$(0.9967032, 0.0043297)$} \\
		& $\bm P_1$ & $(0.994364, 0.0081246)$ & $(0.9937138, 0.0089796)$ & $(0.,9922575 0.0108947)$ & $(0.9902578, 0.0135243)$\\
		& $\bm P_2$ & $(0.9939234, 0.0075452)$ & $(0.9932534, 0.0083741)$ & $(0.9917308, 0.0102022)$ & $(0.9897312, 0.0128318)$\\
		& $\bm P_3$ & $(0.9994791, 0.0013982)$ & $(0.9991609, 0.0018166)$ & $(0.9987924, 0.0023012)$ & $(0.9985727, 0.0025901)$\\
		& $\bm P_4$ & $(0.9990385, 0.0008188)$ & $(0.9987005, 0.0012111)$ & $(0.9982658, 0.0016087)$ & $(0.9980461, 0.0018976)$\\ 
		\midrule
		$5$ & $\bm P'$ & \multicolumn{4}{c}{$(0.9864873, 0.0245632)$}\\
		& $\bm P_1$ & $(0.9811818, 0.0321294)$ & $(0.9801289, 0.0340045)$ & $(0.977917, 0.0377295)$ & $(0.9753326, 0.0425299)$\\
		& $\bm P_2$ & $(0.9827429, 0.0350291)$ & $(0.9817117, 0.0369446)$ & $(0.9796031, 0.0408613)$ & $(0.9770187, 0.0456617)$\\
		& $\bm P_3$ & $(0.9919707, 0.0120896)$ & $(0.9909376, 0.0139279)$ & $(0.9898161, 0.0156277)$ & $(0.9892599, 0.0166608)$\\
		& $\bm P_4$ & $(0.9935319, 0.0149893)$ & $(0.9925204, 0.016868)$ & $(0.9915021, 0.0187595)$ & $(0.9909459, 0.0197926)$\\ 
		\midrule
		$6$ & $\bm P'$ & \multicolumn{4}{c}{$(0,9296429, 0.1841438)$}\\
		& $\bm P_1$ & $(0.9121895, 0.2201993)$ & $(0.9105951, 0.2239294)$ & $(0.9105951, 0.2239294)$ & $(0.9077672, 0.2305874)$\\
		& $\bm P_2$ & $(0.9146048, 0.2260776)$ & $(0.9130413, 0.2298829)$ & $(0.9130413, 0.2298829)$ & $(0.9102595, 0.2366531)$\\
		& $\bm P_3$ & $(0.9443868, 0.1418373)$ & $(0.9419202, 0.1476904)$ & $(0.9396577, 0.1531967)$ & $(0.9383611, 0.1561279)$\\
		& $\bm P_4$ & $(0.9468021, 0.1477156)$ & $(0.9443663, 0.1536439)$ & $(0.9421039, 0.1591502)$ & $(0.9408534, 0.1621937)$\\ 
		\bottomrule
	\end{tabular}}
\end{table}

\begin{table}[hbt]
	\caption{Coordinates in the $H\times C$ plane of the emblematic series and the points that define the confidence regions at \SI{90}{\percent}, \SI{95}{\percent}, \SI{99}{\percent}, and \SI{99.9}{\percent} for $D=3,4,5,6$ and $N=50000$}\label{Tab:Points50k}
	\centering
    \scalebox{0.73}{
	\begin{tabular}{rccccc}
		\toprule
		%& & \multicolumn{4}{c}{$N$}\\ 
		%\cmidrule(lr){3-6}
		& & \multicolumn{4}{c}{$N = 50000$}\\ 
		\cmidrule(lr){3-6} 
		$D$ & Point & \SI{90}{\percent} & \SI{95}{\percent} & \SI{99}{\percent} & \SI{99.9}{\percent}\\ 
		\cmidrule(lr){3-3} 
		\cmidrule(lr){4-4} 
		\cmidrule(lr){5-5} 
		\cmidrule(lr){6-6} 
		$3$ & $\bm P'$ & \multicolumn{4}{c}{$(0.9999853, \num[scientific-notation=true]{1.45e-05})$}\\
		& $\bm P_1$ & $(0.9999489, \num[scientific-notation=true]{5.06e-05})$ & $(0.9999384, \num[scientific-notation=true]{6.11e-05})$ & $(0.9999079, \num[scientific-notation=true]{9.11e-05})$ & $(0.9998625, 0.0001361)$\\
		& $\bm P_2$ & $(0.9999487, \num[scientific-notation=true]{5.04e-05})$ & $(0.9999382, \num[scientific-notation=true]{6.09e-05})$ & $(0.9999077, \num[scientific-notation=true]{9.09e-05})$ & $(0.9998622, 0.0001358)$\\
		& $\bm P_3$ & $(0.9999998, \num[scientific-notation=true]{4e-07})$ & $(0.9999994, \num[scientific-notation=true]{9e-07})$ & $(0.9999982, \num[scientific-notation=true]{2e-06})$ & $(0.9999973, \num[scientific-notation=true]{3e-06})$\\
		& $\bm P_4$ & $(0.9999996, \num[scientific-notation=true]{2e-07})$ & $(0.9999991, \num[scientific-notation=true]{7e-07})$ & $(0.999998, \num[scientific-notation=true]{1.8e-06})$ & $(0.999997, \num[scientific-notation=true]{2.7e-06})$\\ \midrule
		$4$ & $\bm P'$ & \multicolumn{4}{c}{$(0.9999394, \num[scientific-notation=true]{7.94e-05})$}\\
		& $\bm P_1$ & $(0.9999684, \num[scientific-notation=true]{3.98e-05})$ & $(0.9999725, \num[scientific-notation=true]{3.44e-05})$ & $(0.9999783, \num[scientific-notation=true]{2.68e-05})$ & $(0.9999833, \num[scientific-notation=true]{2.02e-05})$ \\
		& $\bm P_2$ & $(0.9999696, \num[scientific-notation=true]{4.13e-05})$ & $(0.9999737, \num[scientific-notation=true]{3.6e-05})$ & $(0.9999795, \num[scientific-notation=true]{2.83e-05})$ & $(0.9999845, \num[scientific-notation=true]{2.18e-05})$ \\
		& $\bm P_3$ & $(0.9998075, 0.0002508)$ & $(0.9998506, 0.0001942)$ & $(0.9998756, 0.0001615)$ & $(0.9998889, 0.000144)$ \\
		& $\bm P_4$ & $(0.9998087, 0.0002524)$ & $(0.9998518, 0.0001958)$ & $(0.9998768, 0.000163)$ & $(0.9998901, 0.0001456)$ \\ \midrule
		$5$ & $\bm P'$ & \multicolumn{4}{c}{$(0.9997616, 0.0004264)$}\\
		& $\bm P_1$ & $(0.9998172, 0.0003232)$ & $(0.9998259, 0.0003075)$ & $(0.9998428, 0.0002774)$ & $(0.9998573, 0.0002517)$ \\
		& $\bm P_2$ & $(0.9998194, 0.0003273)$ & $(0.9998282, 0.0003116)$ & $(0.999845, 0.0002814)$ & $(0.9998593, 0.0002553)$ \\
		& $\bm P_3$ & $(0.9994812, 0.0009246)$ & $(0.9996371, 0.0006455)$ & $(0.9996703, 0.0005862)$ & $(0.9996884, 0.000554)$ \\
		& $\bm P_4$ & $(0.9994834, 0.0009286)$ & $(0.9996394, 0.0006495)$ & $(0.9996725, 0.0005901)$ & $(0.9996904, 0.0005576)$ \\ \midrule
		$6$ & $\bm P'$ & \multicolumn{4}{c}{$(0.9989108, 0.0026093)$}\\
		& $\bm P_1$ & $(0.9990169, 0.002336)$ & $(0.9990368, 0.002288)$ & $(0.9990736, 0.0021997)$ & $(0.9991069, 0.0021197)$ \\
		& $\bm P_2$ & $(0.9990249, 0.0023554)$ & $(0.9990449, 0.0023074)$ & $(0.9990817, 0.0022191)$ & $(0.999115, 0.0021392)$ \\
		& $\bm P_3$ & $(0.9978983, 0.0050219)$ & $(0.998714, 0.0030633)$ & $(0.998765, 0.0029407)$ & $(0.9987884, 0.0028845)$ \\
		& $\bm P_4$ & $(0.9979064, 0.0050413)$ & $(0.998722, 0.0030827)$ & $(0.9987731, 0.0029601)$ & $(0.9987965, 0.0029039)$ \\ \bottomrule
	\end{tabular}}
\end{table}

\subsection{Test Size}

To analyze the efficiency of the confidence region calculated, we tested its applicability on a set of true random data generated physically not used by the algorithm during its construction. 
We assessed the size of the test by contrasting $100$ new TWNRS for each situation of $D=3,4,5,6$ and of $\alpha=0.01,0.05$.
Table~\ref{tab:result1} and Fig.~\ref{fig:RNG} show the results.

On the one hand, long series ($T=50000$) present a good size for every embedding dimension.
On the other hand, short series ($T=1000$) exhibit only one situation with a noticeable divergence between the expected and the observed size: the test rejects \SI{13}{\percent} of the $100$ series when $D=6$. 
In contrast, we expected \SI{1}{\percent} of rejection.
This might be because, in this case, the condition $D!\ll T$ is not respected.
Notice that the wrongly rejected TWNRS are all close to the point $(1,0)$.

We may then conclude that the test has good empirical size, provided $D!\ll T$, a condition that does not hold for $D=6$ and $T=1000$.

\begin{figure}
    \centering
    \includegraphics[width=\linewidth]{Figures/RNG-1000.pdf}
    \includegraphics[width=\linewidth]{Figures/RNG-50000.pdf}
    \caption{Results of the analysis behavior of true random noises in the regions of confidence built.}
    \label{fig:RNG}
\end{figure}

\subsection{Test Power}

We assessed the power of the test by contrasting time series with different correlation structure (under the $f^{-k}$ model) in the $H \times C$ plane.
Several studies in the literature have used this approach for identifying and characterizing randomness.

Our study's basis is the emblematic time series for each length $T$ and dimension embedding $D$.
Recall that the emblematic time was chosen as the most representative of the data set.
We use these series, transform them into $f^{-k}$ correlated noise, and verify the new point's location in the $H\times C$ plane.

As we can observe in the plane, as the correlation between the observations increases, that is, $k > 0$, the randomness decreases, and the entropy presented decreases, informing the loss of its stochastic characteristic.

Fig.~\ref{fig:CorrNoisea} shows the overall effect of transforming the emblematic time series into $f^{-k}$ correlated noise, with $k=1/2,1,3/2,2,5/2,3$.
At this scale, the emblematic time series $k=0$ and the one with $k=1/2$ appear overlapped.
As the correlation increases with $k$, the randomness decreases, causing a drop in the entropy; the series become progressively more predictable.

Fig.~\ref{fig:CorrNoiseb} is a zoom close to the $(1,0)$ point, along with the confidence regions for the white noise.
We see that $ k = 0 $ and $ k = 0.1 $ are inside the $\SI{95}{\percent}$ confidence region, and $ k = 0.2 $ is inside the $\SI{99}{\percent}$ box.
Notice that the time series with $k=3/10$ is outside the confidence regions and does not pass the randomness test.
The same holds for all $k>3/10$.

\begin{figure*}
	\centering
	\subfigure[Points in the $H\times C$ plane of the emblematic white noise  ($k=0$) and its transformations to become $f^{-k}$ correlated noise with $k=1/2,1,3/2,2,5/2,3$.\label{fig:CorrNoisea}]{
		\includegraphics[width=.48\linewidth]{Figures/Correlation-Analysis-dotted.pdf}}
	\subfigure[Points of the emblematic white noise ($k=0$), and its $f^{-k}$ correlated noise versions, with $k=1/10, 1/5,3/10$ along with the confidence regions for white noise.\label{fig:CorrNoiseb}]{
		\includegraphics[width=.48\linewidth]{Figures/Correlation-Analysis-point.pdf}}
	\caption{Analysis of the test power with correlated $f^{-k}$ noise.}
	\label{fig:correlation}
\end{figure*}

\subsection{Revisiting the White Noise Hypothesis in the Literature}

In this section, we compare the performance of our test with that of 
previous analyses that employ the Complexity-Entropy plane.
To this aim, we produced $100$ sequences of length $T = \num[scientific-notation=true]{5 e4}$ for each generator and computed the $p$-value for each $D = \{3, 4, 5, 6\}$.
Previous results are shown in Table~\ref{Tab:Literature}, and ours are in Table~\ref{Tab:LiteratureComparations}.
We grouped our results in those that rejected (R) the null hypothesis and those that did not reject it (NR).

Comparing Tables~\ref{Tab:Literature} and~\ref{Tab:LiteratureComparations}, we see that our test captures adequately the random dynamics of the sequences produced by most of the analyzed generators.
It is noteworthy that the generator Combo RNG sequences only pass our white noise test for $D = 3$.
In higher embedding dimensions, as we consider longer words, the sequences produced by this generator are not labeled as white noise.

\begin{table}
	\caption{Results of the sequences generated by the main PRNGs in the literature. 
		The sequences have length $T=\num[scientific-notation = true]{5 e4}$.}
	\label{Tab:LiteratureComparations}
	\centering
    \scalebox{0.99}{
	\begin{tabular}{cccc}
		\toprule
		Algorithm & 
		\multicolumn{1}{c}{$D$} & 
		$p$-value &
		HC-PCA\\
		\cmidrule(lr){1-1}
		\cmidrule(lr){2-2}
		\cmidrule(lr){3-3}
		\cmidrule(lr){4-4}
		MOT & 3 & 0.305 & NR\\
		& 4 & 0.572 & NR\\ 
		& 5 & 0.455 & NR\\ 
		& 6 & 0.508 & NR\\ 
		\cmidrule(lr){1-4}
		MWC & 3 & 0.501 & NR\\
		& 4 & 0.477 & NR\\ 
		& 5 & 0.496 & NR\\ 
		& 6 & 0.496 & NR\\ 
		\cmidrule(lr){1-4}
		COM & 3 & 0.123 & NR\\
		& 4 & 0.002 & R\\ 
		& 5 & \num[scientific-notation=true]{1.11 e-16} & R\\ 
		& 6 & \num[scientific-notation=true]{1.11 e-16} & R\\ 
		\cmidrule(lr){1-4}
		LEH & 3 & 0.531 & NR\\
		& 4 & 0.515 & NR\\ 
		\bottomrule
	\end{tabular}
	\begin{tabular}{|cccc}
		\toprule
		Algorithm & 
		\multicolumn{1}{c}{$D$} & 
		$p$-value &
		HC-PCA\\
		\cmidrule(lr){1-1}
		\cmidrule(lr){2-2}
		\cmidrule(lr){3-3}
		\cmidrule(lr){4-4}
		LEH & 5 & 0.495 & NR\\ 
		& 6 & 0.501 & NR\\ 
		\cmidrule(lr){1-4}
		fGn & 3 & 0.521 & NR\\
		& 4 & 0.519 & NR\\ 
		& 5 & 0.498 & NR\\ 
		& 6 & 0.470 & NR\\
		\cmidrule(lr){1-4}
		$f^{-k}$ & 3 & 0.482 & NR\\
		& 4 & 0.520 & NR\\ 
		& 5 & 0.513 & NR\\ 
		& 6 & 0.508 & NR\\
		\cmidrule(lr){1-4}
		LCG & 3 & 0.009 & R\\ 
		& 4 & \num[scientific-notation=true]{1.11 e-16} & R\\ 
		& 5 & \num[scientific-notation=true]{1.11 e-16} & R\\ 
		& 6 & \num[scientific-notation=true]{1.11 e-16} & R\\ 
		\bottomrule
	\end{tabular}}
\end{table}

\section{Conclusions}\label{Sec:Conclusions}

We presented and evaluated the first test for white noise in the  Complexity-Entropy plane.
Our proposal is based on two stages:
(1)~building non-parametric empirical confidence regions in the principal components space and mapping these boxes back to the $H\times C$ plane.
(2)~computing an approximate $p$-value for a given sample by comparing it with points produced by true white noise random sequences (TWNRS).
We obtained the TWNRS with data from physical devices.

Our test has a good size, mostly with long TWNRS.
We also determined the power of our test for the alternative hypothesis of correlated $f^{-k}$ noise and found that it rejects the null hypothesis ($k=0$) for $k>3/10$.

Although our work focuses on the study of short sequences, we were able to capture the random behavior of well-known pseudorandom number generators already analyzed in the literature. 
With this, we verified the adequacy of our technique as it is capable of detecting correlation structures.

\section{Reproducibility and Replicability}\label{Sec:code}

Following the recommendations provided by \cite{ABadgingSystemforReproducibilityandReplicabilityinRemoteSensingResearch}, we make the text, source code, and data used in this study available at the \textit{Confidence-Regions} repository \url{https://github.com/EduardaChagas/ConfidenceRegions}.
