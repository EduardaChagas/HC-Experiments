\documentclass{article}

\usepackage{spconf,amsmath}
\usepackage{bm,bbm}
\usepackage[boxed]{algorithm2e}
\usepackage[caption=false,font=footnotesize]{subfig}
\usepackage[binary-units]{siunitx}
\usepackage{booktabs}
\usepackage{url}
\usepackage{graphicx}
\usepackage[numbers]{natbib}


\title{\textbf{An Exploratory Study of the Transition Graphs Application in SAR Textures}}

\name{Eduarda T.\ C.\ Chagas$^a$, Heitor S.\ Ramos$^a$, Osvaldo A.\ Rosso$^b$, and Alejandro C.\ Frery$^c$}
\address{
	$^a$Departamento de Ci\^encia da Computa\c c\~ao, Universidade Federal de Minas Gerais, Brazil
	\\
	$^b$Instituto de F\'isica, Universidade Federal de Alagoas, Brazil
	\\
	$^c$School of Mathematics and Statistics, Victoria University of Wellington, New Zealand
}

\begin{document}
	\maketitle
	
	\begin{abstract}
		
		A new perspective on the analysis of surfaces and land use has been carried out with the synthetic aperture images (SAR) extraction and classification.
		With the popularization of the machine and deep learning techniques, such algorithms have revolutionized the performance and interpretability of their results~\cite{han2020unsupervised, huang2020classification, xie2020polsar}.
		
		On the other hand, in the context of non-parametric time series analysis, the methodology of~\citet{PermutationEntropyBandtPompe} has been used successfully in several scientific areas~\cite{baravalle2018discriminating, Araujo2019permutation, ClassificationVerificationOnlineHandwrittenSignatures}.
		Obtaining a new representation of the time series through the ordinal patterns, the distributions resulting from the histogram of frequencies or the pattern transition graphs become less sensitive to outliers and do not assume any hypothesis about the nature of the data.
		Despite its simplicity, this method is robust to noise, has a low computational cost and when used in conjunction with causal descriptors from the Information Theory, it has been showing good results in the signals characterization and classification.
		
		In this context, the work developed by~\citet{ChagasClassification2020} proposes an applicability expansion of this methodology in the remote sensing context, presenting a new perspective of extracting characteristics in textures from SAR images.
		For that, it was necessary to perform the dimensionality reduction transforming the patch of homogeneous textures \mbox{2-D} into a signal \mbox{1-D} through the Hilbert-Peano curves~\cite{Lee1994Texture} .
		With this approach, it was found that it is possible to preserve relevant properties of spatial pixel correlation (since such a curve never maintains the same orientation for more than three consecutive points) with a low computational cost.
		
		The main contribution consists of the new ordinal patterns transition graphs, called weighted amplitude transition graph (WATG).
		With this graph, it was able to discriminate similar patterns with different intensity variations.
		Here, we obtain the probability distribution for the ordinal pattern transitions along the sequence.
		It is through the  Information Theory descriptors that we extract important characteristics of the underlying process and we obtaining interpretability of the observations in terms of spatial dependence.
		These descriptors are Shannon's Entropy and Statistical Complexity.
		
		The first descriptor is a measure of the system's disorder, and is defined as
		\begin{equation}
			H(\mathbbm{P}) = -\frac{1}{\log D!} \sum_{\ell=1}^{D!^2} \Big(p_\ell \log p_\ell\Big),
		\end{equation}
		where $D$ represents the ordinal patterns dimension used in the symbolization process, and 
		$$\mathbbm{P} = \{p_{(\widetilde\pi^D_1, \widetilde\pi^D_1)}, p_{(\widetilde\pi^D_1, \widetilde\pi^D_2)}, \dots, p_{(\widetilde\pi^D_{D!}, \widetilde\pi^D_{D!})} \} = \{p_1,\dots,p_{D!^2}\}$$
		is the probability distribution obtained of the image patch when WATG is applied.
		
		Although very expressive, the Normalized Shannon Entropy is not able to describe all possible underlying dynamics.
		To this aim, \citet{LopezRuiz1995} proposed using the disequilibrium  $Q$, a measure of how far $\mathbbm{P}$ is from an equilibrium or non-informative distribution $\mathbbm{U}$.
		We calculated this descriptor as:
		\begin{equation}
			Q'(\mathbbm{P}, \mathbbm{U}) = \sum_{\ell=1}^{D!^2} \Big(p_\ell \log\frac{p_\ell}{u_\ell} +
			u_\ell \log\frac{u_\ell}{p_\ell}
			\Big),
		\end{equation}
		where our normalized formula is $Q = Q'/\max\{Q'\}$.
		With this, is proposed the Statistical Complexity $C = HQ$ which measures the dependence structures among the elements.
		We can map a sequence at a point $(h, c)$, where the set of all possible points is the Entropy-Complexity plane (or $H \times C$ plane).
		
		\begin{figure*}
			\includegraphics[width=\linewidth]{SARTexture.pdf}
			\caption{Outline of the methodology used for the textures classification.}
			\label{fig:Outline}
		\end{figure*} 
		
		
		When we represent each fragment of UAVSAR images as a point in the Entropy complexity plane, we achieve a perfect separation between urban, pastures, ocean, and forest areas.
		In addition to presenting a better performance than the analyzed baselines, this characterization has some advantages:
		~(i) provides easy visualization of features and
		~(ii) your training process with machine learning algorithms becomes faster and less costly.
		
		The intensity of the SAR texture signals is directly related to the analyzed target's nature and backscattering properties.
		Therefore, using consolidated techniques in the non-parametric time series analysis to study the spatial structure of the pixels represents a fruitful field of investigation.
		
		In this context, we seek to analyze WATG in new scenarios for the classification of homogeneous textures, in which the following research questions stand out:
		\begin{enumerate}
			\item What is the impact of the dimensions variation of Hilbert curves on the final descriptors?
			When characterizing and classifying SAR images, the application of Hilbert-Peano curves with dimensions equal to $128 \times 128$ is suggested.
			Thus, the objective of this experiment is to analyze the minimum dimension necessary to obtain a good characterization when we apply WATG in textures of SAR images;
			
			\item Can we observe variations in the results when we apply other backscatter bands?
			Since WATG is investigated only with polarimetric SAR quad images with HHHH backscatter magnitudes, here we investigated its use in other magnitudes and their impact on the HC plane;
			
			\item How does the algorithm behave when exposed to different surfaces?
			For this, we selected new L-band polarimetric SAR images with HHHH backscatter magnitudes from the unmanned aerial vehicle synthetic radar sensor (UAVSAR) of the NASA Jet Propulsion Laboratory (JPL).
			We built a dataset with $700$ image patches and with the same dimensions used in the previous work for Hilbert-Peano curves, containing the following classes: pasture, forest, oceans, glacial, desert, and urban regions.
		\end{enumerate}
		
	\end{abstract}
	
	\keywords{
		Synthetic Aperture Radar (SAR), 
		Terrain Classification,		
		Information Theory, 
		Ordinal Patterns.
	}
	
	\bibliographystyle{IEEEtranSN}
	\bibliography{ref}
	
\end{document}